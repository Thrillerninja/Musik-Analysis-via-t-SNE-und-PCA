\chapter{Installation}

\section{Setup der PostgreSQL-Datenbank mit pgvector}

Für die Speicherung und Analyse von Musikdaten verwenden wir eine PostgreSQL-Datenbank mit der \texttt{pgvector}-Erweiterung. Diese ermöglicht die effiziente Speicherung und Abfrage hochdimensionaler Vektoren. Im Folgenden wird beschrieben, wie die Datenbank mit Docker eingerichtet wird.

\subsection{Docker-Setup}

Wir verwenden Docker, um die PostgreSQL-Datenbank mit der \texttt{pgvector}-Erweiterung zu starten. Die Konfiguration erfolgt über die folgende \texttt{docker-compose.yml}-Datei:

\lstinputlisting[
  language=yaml, 
  caption={Unkomplizierte docker-compose.yml Datei}, 
  basicstyle=\small\ttfamily, 
]{../compose.yml}


\paragraph{Schritte zur Ausführung:}
\begin{enumerate}
    \item Speichern Sie die obige Konfiguration in einer Datei mit dem Namen \texttt{docker-compose.yml}.
    \item Starten Sie den PostgreSQL-Container mit dem Befehl:
    \begin{lstlisting}[language=bash]
    docker-compose up -d
    \end{lstlisting}
    \item Überprüfen Sie, ob der Container läuft:
    \begin{lstlisting}[language=bash]
    docker ps
    \end{lstlisting}
\end{enumerate}

\subsection{Initialisierung der Datenbank}

Nach dem Start der PostgreSQL-Datenbank muss die \texttt{pgvector}-Erweiterung aktiviert und die Datenbankstruktur eingerichtet werden. Dafür wurde dieses SQL Skript erstellt \texttt{init.sql}:

\lstinputlisting[
    language=sql, 
    caption={Setup int.sql Datei}, 
    basicstyle=\small\ttfamily, 
]{../init.sql}

Dieses Skript erstellt eine Tabelle \texttt{music\_vectors} mit einer Spalte \texttt{vector} vom Typ \texttt{vector}. Die \texttt{pgvector}-Erweiterung wird aktiviert und die Tabelle wird mit einem IVFFLAT-Index für die Vektorspalte erstellt. 

Speichern Sie die obige SQL-Datei als \texttt{init.sql}, die Ausführung wird durch das Hautprogramm, dem angehängten \texttt{jupyter notebook}, durchgeführt.

\section{Herunterladen der genutzten Musikdaten}

Die Musikdaten, die in diesem Projekt verwendet werden, sind in der \texttt{spotify\_dataset.csv}-Datei enthalten, welche von \href{https://www.kaggle.com/datasets/bricevergnou/spotify-recommendation}{Kaggle} heruntergeladen wurde. Diese Datei enthält Eigenschaften von verschiedenen Musikstücken, die Informationen wie die Lautstärke, die Energie, die Tanzbarkeit und die Dauer enthalten.